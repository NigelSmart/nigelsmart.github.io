\documentstyle[amscd,12pt,myidx,psfig]{mybook}
\textheight 215mm
\textwidth 138mm
\markboth{\sc elliptic curves in cryptography}{\sc errata}
%
%  Macro for eps figures
%
\newcommand{\psXfigure}[4]{
    \begin{figure}[tp]
    \centerline{
        \hbox{\psfig{figure=#1,width=#2}}
    }
    \caption{\label{#4}#3}
    \end{figure}
}

\newtheorem{theorem}{Theorem}[chapter]
\newtheorem{lemma}[theorem]{Lemma}
\newtheorem{cor}[theorem]{Corollary}
\newtheorem{corollary}[theorem]{Corollary}
\newtheorem{prop}[theorem]{Proposition}
\newtheorem{conjecture}[theorem]{Conjecture}
\newtheorem{example}{Example}
\newtheorem{defn}[theorem]{Definition}

%Numbering
\renewcommand{\thechapter}{\Roman{chapter}}
\renewcommand{\thetheorem}{{\thechapter}.\arabic{theorem}}
\renewcommand{\thelemma}{{\thechapter}.\arabic{lemma}}
\renewcommand{\thecor}{{\thechapter}.\arabic{cor}}
\renewcommand{\theprop}{{\thechapter}.\arabic{prop}}
\renewcommand{\theconjecture}{{\thechapter}.\arabic{conjecture}}
\renewcommand{\theexample}{\arabic{example}}
\renewcommand{\thesection}{{\thechapter}.\arabic{section}}
\renewcommand{\thesubsection}
      {{\thechapter}.\arabic{section}.\arabic{subsection}}
\renewcommand{\theequation}{{\thechapter}.\arabic{equation}}
\renewcommand{\thefigure}{{\thechapter}.\arabic{figure}}
\renewcommand{\thetable}{{\thechapter}.\arabic{table}}

%Math symbols used
\newcommand{\ord}{{\rm ord}}
\newcommand{\lcm}{{\rm lcm}}
\newcommand{\adj}{{\rm adj}}
\newcommand{\dv}{{\rm div}}
\newcommand{\DLog}{{\rm DL}}
\newcommand{\DHP}{{\rm DHP}}

\newcommand{\C}{{\Bbb C}}
\newcommand{\D}{{\Bbb D}}
\newcommand{\E}{{\Bbb E}}
\newcommand{\F}{{\Bbb F}}
\newcommand{\G}{{\Bbb G}}
\newcommand{\N}{{\Bbb N}}
\newcommand{\PP}{{\Bbb P}}
\newcommand{\Q}{{\Bbb Q}}
\newcommand{\R}{{\Bbb R}}
\newcommand{\Z}{{\Bbb Z}}

\newcommand{\EE}{{\cal E}}
\newcommand{\LL}{{\cal L}}
\newcommand{\OO}{{\cal O}}
\newcommand{\HH}{{\cal H}}
\newcommand{\II}{{\cal I}}
\newcommand{\GLL}{{\G_{\LL}}}
\newcommand{\FF}{{\cal F}}

\newcommand{\ga}{{\frak{a}}}
\newcommand{\gb}{{\frak{b}}}
\newcommand{\gp}{{\frak{p}}}
\newcommand{\gq}{{\frak{q}}}
\newcommand{\gB}{{\frak{B}}}
\newcommand{\gS}{{\frak{S}}}
\newcommand{\hash}{{\#}}

\newcommand{\floor}[1]{\lfloor #1 \rfloor}
\newcommand{\ceil}[1]{\lceil #1 \rceil}
\newcommand{\ceilsqrt}[1]{\lceil\sqrt{#1}\, \rceil}
\newcommand{\nth}{{\hat{h}}}
\newcommand{\grp}[1]{{\langle #1 \rangle}}

%
% Newcommands added by GS
%
\newcommand{\fq}{\F_q}
\newcommand{\fqn}{\F_{q^n}}
\newcommand{\fp}{{\F}_p}
\newcommand{\fqstar}{\F_q^{\,\ast}}
\newcommand{\fqlstar}{\F_{q^l}^{\,\ast}}
\newcommand{\fqlstarn}{(\fqlstar)^n}
\newcommand{\fpstar}{{\F}_p^{\,\ast}}
\newcommand{\fbp}{{\overline{\F}}_p}
\newcommand{\fbq}{{\overline{\F}}_q}
\newcommand{\fbtn}{\overline{\F}_{2^n}}
\newcommand{\fbpstar}{{\overline{\F}}_p^{\,\ast}}
\newcommand{\fbqstar}{{\overline{\F}}_q^{\,\ast}}
\newcommand{\fl}{{\F}_{\ell}}
\newcommand{\ft}{\mbox{$\F_2$}}
\newcommand{\ftn}{\mbox{$\F_{2^n}$}}
\newcommand{\tr}{\mbox{\rm Tr}}
\newcommand{\trq}{\mbox{\rm Tr}_{q|2}}
\newcommand{\trtn}{\mbox{\rm Tr}_{q|2}}
\newcommand{\window}[1]{\,\underline{#1}\,}
\newcommand{\jt}{\tilde{\jmath}\:}
\newcommand{\Kstar}{K^{\,\ast}}
\newcommand{\Kbar}{\overline{K}}
\newcommand{\Kbstar}{\overline{K}^{\,\ast}}
\newcommand{\chr}{\mbox{char}}
\newcommand{\psib}{\overline{\psi}}
\newcommand{\ql}{q_{\ell}}
\newcommand{\fbar}{\overline{f}}

\newcommand{\frob}{\varphi}
\newcommand{\totient}{\phi_{\mbox{\scriptsize Eul}}}
\newcommand{\isog}{\phi}
\newcommand{\frobchl}{\FF_\ell}
\newcommand{\padiclog}{\vartheta}
\newcommand{\bino}[2]{B(#1,#2)}
\newcommand{\Eb}{\overline{E}}

\newcommand{\SL}{\mbox{\it SL}}
\newcommand{\GL}{\mbox{\it GL}}
\newcommand{\PGL}{\mbox{\it PGL}}
\newcommand{\cec}{C_{\mbox{\tiny EC}}}
\newcommand{\cconv}{C_{\mbox{\tiny CONV}}}

%\newcommand{\DF}{\mbox{\em DF}}
%\newcommand{\DFl}{\mbox{\em DF}_2}
\newcommand{\DF}{D_g}
\newcommand{\DFl}{D_g^*}
%\newcommand{\DJ}{\mbox{\em DJ}}
%\newcommand{\DJl}{\mbox{\em DJ}_2}
\newcommand{\DJ}{D_j}
\newcommand{\DJl}{D_j^*}

\newcommand{\wt}{W}
\newcommand{\Jc}{J_C}
\newcommand{\Jcfq}{J_C(\fq)}
\newcommand{\Jcfqn}{J_C(\fqn)}
\newcommand {\eqdef} {\stackrel{\Delta}{=}}

\newcommand{\End}{\mbox{\rm End}}
\newcommand{\Gal}{\mbox{\rm Gal}}
\newcommand{\Aut}{\mbox{\rm Aut}}

\newcommand{\regtmark}{$\!^{\bigcirc \!\!\!\!\!\mbox{\rm \tiny R}}$}

%
% Redefine built-in macro \pmod to produce more pleasing spacing
%
\renewcommand{\pmod}[1]{\;(\mbox{\rm mod}\;#1)}


%
% DEFs FOR ALGORITHMS (LaTeX version)
%
%    (DONATED BY JOHN CREMONA, MODIFIED BY NPS.
%    Nested line numbers, reference labels, other minor
%    modifications added by GS.
%
%

% Define 4 levels of nesting
\newcounter{lineno}
\newcounter{linelineno}[lineno]
\newcounter{linelinelineno}[linelineno]
\newcounter{linelinelinelineno}[linelinelineno]

\newcounter{algnum}[chapter]
\renewcommand{\thealgnum}{\thechapter.\arabic{algnum}}

% Algorithm as new environment
\newenvironment{algorithm}[3]
{
        \refstepcounter{algnum}
        \bigskip\goodbreak\hrule\medskip
                \leftline{{\sc Algorithm \thealgnum:} \bf #1}
                \medskip\hrule\medskip\nobreak
                \tt
        \begin{tabbing}
        OUTPUT: \= \kill
        INPUT:  \>#2 \\
        OUTPUT: \>#3
        \end{tabbing}
                \begin{tabbing}
%This sets tab positions: might need adjustment!
10.\ \=word\=word\=word\=word\=word\=word\=\kill
                \smallskip \setcounter{lineno}{0}
}
{           \end{tabbing}
            \par\nobreak\hrule\bigskip
            \rm
}

% New line in algorithm
\newcommand{\nline}{\refstepcounter{lineno} %
\'\thelineno.\>%
}
% New line with referable label
\newcommand{\nlinelbl}[1]{\refstepcounter{lineno}\label{#1}%
\'\thelineno.\>%
}

% New line at second level
\renewcommand{\thelinelineno}{\thelineno.\arabic{linelineno}}
\newcommand{\nnline}{\refstepcounter{lineno}%
\'\thelineno.\>\>%
}

% New line (2nd level) with referable label
\newcommand{\nnlinelbl}[1]{\refstepcounter{lineno}\label{#1}%
\'\thelineno.\>\>%
}

% New line at third level
\renewcommand{\thelinelinelineno}{\thelinelineno.\arabic{linelinelineno}}
\newcommand{\nnnline}{\refstepcounter{lineno}%
\'\thelineno.\>\>\>%
}

% New line (3rd level) with referable label
\newcommand{\nnnlinelbl}[1]{\refstepcounter{lineno}\label{#1}%
\'\thelineno.\>\>\>%
}

% New line at 4th level
\newcommand{\nnnnline}{\refstepcounter{lineno}%
\'\thelineno.\>\>\>\>%
}

% New line (4th level) with referable label
\newcommand{\nnnnlinelbl}[1]{\refstepcounter{lineno}\label{#1}%
\'\thelineno.\>\>\>\>%
}




% Put a comment line in an algorithm, no line number.
\def\comment{\'\em }


%\settabs 17\columns

% Define the "assignment" symbol for algorithms
\newcommand{\asn}{\mbox{$\,\leftarrow$\,}}

%
% END OF DEFs FOR ALGORITHMS
%


\begin {document}
\begin{center}

\vspace{0.5in}

{\Large\em Elliptic Curves in Cryptography}

\vspace{0.25in}
{\large Ian Blake, Gadiel Seroussi and Nigel Smart}

\vspace{0.5in}

{\Large\bf Errata}

(for second edition and printings of June 2000 and later)

\end{center}

\vspace{0.15in}

%\addtolength{\parskip}{5mm}

\begin{description}
\addtolength{\itemsep}{2mm}

\item[p.\ 9] The formula for $\cconv(N)$ has a $\log 2$ missing. It
should read
\[
\cconv(N) = \exp\left(\, c_0 \,(N \log 2)^{1/3} \,(\log (N \log 2) )^{2/3} \,\right).
\]

\item[p.\ 47] Line $-1$. Replace ``$a=-g_2/\sqrt[3]{4}$, $b=-g_3$'' with
  ``$a=-g_2/4$, $b=-g_3/4$''.

\item[p.\ 54] Line $11$. Replace ``for $i=1, ..., \ell + 1$'' should be
 ``for $r=1, ..., \ell + 1$''.
\end{description}

\end {document}
