\documentstyle[amscd,12pt,myidx,psfig]{mybook}
\textheight 215mm
\textwidth 138mm
\markboth{\sc elliptic curves in cryptography}{\sc errata} 
%
%  Macro for eps figures
%
\newcommand{\psXfigure}[4]{
	\begin{figure}[tp]
	\centerline{
	    \hbox{\psfig{figure=#1,width=#2}}
	}
	\caption{\label{#4}#3}
	\end{figure}
}

\newtheorem{theorem}{Theorem}[chapter]
\newtheorem{lemma}[theorem]{Lemma}
\newtheorem{cor}[theorem]{Corollary}
\newtheorem{corollary}[theorem]{Corollary}
\newtheorem{prop}[theorem]{Proposition}
\newtheorem{conjecture}[theorem]{Conjecture}
\newtheorem{example}{Example}
\newtheorem{defn}[theorem]{Definition}

%Numbering
\renewcommand{\thechapter}{\Roman{chapter}}
\renewcommand{\thetheorem}{{\thechapter}.\arabic{theorem}}
\renewcommand{\thelemma}{{\thechapter}.\arabic{lemma}}
\renewcommand{\thecor}{{\thechapter}.\arabic{cor}}
\renewcommand{\theprop}{{\thechapter}.\arabic{prop}}
\renewcommand{\theconjecture}{{\thechapter}.\arabic{conjecture}}
\renewcommand{\theexample}{\arabic{example}}
\renewcommand{\thesection}{{\thechapter}.\arabic{section}}
\renewcommand{\thesubsection}
      {{\thechapter}.\arabic{section}.\arabic{subsection}}
\renewcommand{\theequation}{{\thechapter}.\arabic{equation}}
\renewcommand{\thefigure}{{\thechapter}.\arabic{figure}}
\renewcommand{\thetable}{{\thechapter}.\arabic{table}}

%Math symbols used
\newcommand{\ord}{{\rm ord}}
\newcommand{\lcm}{{\rm lcm}}
\newcommand{\adj}{{\rm adj}}
\newcommand{\dv}{{\rm div}}
\newcommand{\DLog}{{\rm DL}}
\newcommand{\DHP}{{\rm DHP}}

\newcommand{\C}{{\Bbb C}}
\newcommand{\D}{{\Bbb D}}
\newcommand{\E}{{\Bbb E}}
\newcommand{\F}{{\Bbb F}}
\newcommand{\G}{{\Bbb G}}
\newcommand{\N}{{\Bbb N}}
\newcommand{\PP}{{\Bbb P}}
\newcommand{\Q}{{\Bbb Q}}
\newcommand{\R}{{\Bbb R}}
\newcommand{\Z}{{\Bbb Z}}

\newcommand{\EE}{{\cal E}}
\newcommand{\LL}{{\cal L}}
\newcommand{\OO}{{\cal O}}
\newcommand{\HH}{{\cal H}}
\newcommand{\II}{{\cal I}}
\newcommand{\GLL}{{\G_{\LL}}}
\newcommand{\FF}{{\cal F}}

\newcommand{\ga}{{\frak{a}}}
\newcommand{\gb}{{\frak{b}}}
\newcommand{\gp}{{\frak{p}}}
\newcommand{\gq}{{\frak{q}}}
\newcommand{\gB}{{\frak{B}}}
\newcommand{\gS}{{\frak{S}}}
\newcommand{\hash}{{\#}}

\newcommand{\floor}[1]{\lfloor #1 \rfloor}
\newcommand{\ceil}[1]{\lceil #1 \rceil}
\newcommand{\ceilsqrt}[1]{\lceil\sqrt{#1}\, \rceil}
\newcommand{\nth}{{\hat{h}}}
\newcommand{\grp}[1]{{\langle #1 \rangle}}

%
% Newcommands added by GS
%
\newcommand{\fq}{\F_q}
\newcommand{\fqn}{\F_{q^n}}
\newcommand{\fp}{{\F}_p}
\newcommand{\fqstar}{\F_q^{\,\ast}}
\newcommand{\fqlstar}{\F_{q^l}^{\,\ast}}
\newcommand{\fqlstarn}{(\fqlstar)^n}
\newcommand{\fpstar}{{\F}_p^{\,\ast}}
\newcommand{\fbp}{{\overline{\F}}_p}
\newcommand{\fbq}{{\overline{\F}}_q}
\newcommand{\fbtn}{\overline{\F}_{2^n}}
\newcommand{\fbpstar}{{\overline{\F}}_p^{\,\ast}}
\newcommand{\fbqstar}{{\overline{\F}}_q^{\,\ast}}
\newcommand{\fl}{{\F}_{\ell}}
\newcommand{\ft}{\mbox{$\F_2$}}
\newcommand{\ftn}{\mbox{$\F_{2^n}$}}
\newcommand{\tr}{\mbox{\rm Tr}}
\newcommand{\trq}{\mbox{\rm Tr}_{q|2}}
\newcommand{\trtn}{\mbox{\rm Tr}_{q|2}}
\newcommand{\window}[1]{\,\underline{#1}\,}
\newcommand{\jt}{\tilde{\jmath}\:}
\newcommand{\Kstar}{K^{\,\ast}}
\newcommand{\Kbar}{\overline{K}}
\newcommand{\Kbstar}{\overline{K}^{\,\ast}}
\newcommand{\chr}{\mbox{char}}
\newcommand{\psib}{\overline{\psi}}
\newcommand{\ql}{q_{\ell}}
\newcommand{\fbar}{\overline{f}}

\newcommand{\frob}{\varphi}
\newcommand{\totient}{\phi_{\mbox{\scriptsize Eul}}}
\newcommand{\isog}{\phi}
\newcommand{\frobchl}{\FF_\ell}
\newcommand{\padiclog}{\vartheta}
\newcommand{\bino}[2]{B(#1,#2)}
\newcommand{\Eb}{\overline{E}}

\newcommand{\SL}{\mbox{\it SL}}
\newcommand{\GL}{\mbox{\it GL}}
\newcommand{\PGL}{\mbox{\it PGL}}
\newcommand{\cec}{C_{\mbox{\tiny EC}}}
\newcommand{\cconv}{C_{\mbox{\tiny CONV}}}

%\newcommand{\DF}{\mbox{\em DF}}
%\newcommand{\DFl}{\mbox{\em DF}_2}
\newcommand{\DF}{D_g}
\newcommand{\DFl}{D_g^*}
%\newcommand{\DJ}{\mbox{\em DJ}}
%\newcommand{\DJl}{\mbox{\em DJ}_2}
\newcommand{\DJ}{D_j}
\newcommand{\DJl}{D_j^*}

\newcommand{\wt}{W}
\newcommand{\Jc}{J_C}
\newcommand{\Jcfq}{J_C(\fq)}
\newcommand{\Jcfqn}{J_C(\fqn)}
\newcommand {\eqdef} {\stackrel{\Delta}{=}}

\newcommand{\End}{\mbox{\rm End}}
\newcommand{\Gal}{\mbox{\rm Gal}}
\newcommand{\Aut}{\mbox{\rm Aut}}

\newcommand{\regtmark}{$\!^{\bigcirc \!\!\!\!\!\mbox{\rm \tiny R}}$}

%
% Redefine built-in macro \pmod to produce more pleasing spacing
%
\renewcommand{\pmod}[1]{\;(\mbox{\rm mod}\;#1)}


%
% DEFs FOR ALGORITHMS (LaTeX version) 
%
%    (DONATED BY JOHN CREMONA, MODIFIED BY NPS.
%    Nested line numbers, reference labels, other minor
%    modifications added by GS.
%    
%

% Define 4 levels of nesting
\newcounter{lineno}
\newcounter{linelineno}[lineno]
\newcounter{linelinelineno}[linelineno]
\newcounter{linelinelinelineno}[linelinelineno]

\newcounter{algnum}[chapter]
\renewcommand{\thealgnum}{\thechapter.\arabic{algnum}}

% Algorithm as new environment
\newenvironment{algorithm}[3]
{
		\refstepcounter{algnum}
		\bigskip\goodbreak\hrule\medskip
                \leftline{{\sc Algorithm \thealgnum:} \bf #1}
                \medskip\hrule\medskip\nobreak
                \tt
		\begin{tabbing}
		OUTPUT: \= \kill
		INPUT:  \>#2 \\
		OUTPUT: \>#3
		\end{tabbing}
                \begin{tabbing}
%This sets tab positions: might need adjustment!
10.\ \=word\=word\=word\=word\=word\=word\=\kill
                \smallskip \setcounter{lineno}{0}
}
{           \end{tabbing}
            \par\nobreak\hrule\bigskip
            \rm
}

% New line in algorithm
\newcommand{\nline}{\refstepcounter{lineno} %
\'\thelineno.\>%
}
% New line with referable label
\newcommand{\nlinelbl}[1]{\refstepcounter{lineno}\label{#1}%
\'\thelineno.\>%
}

% New line at second level
\renewcommand{\thelinelineno}{\thelineno.\arabic{linelineno}}
\newcommand{\nnline}{\refstepcounter{lineno}%
\'\thelineno.\>\>%
}

% New line (2nd level) with referable label
\newcommand{\nnlinelbl}[1]{\refstepcounter{lineno}\label{#1}%
\'\thelineno.\>\>%
}

% New line at third level
\renewcommand{\thelinelinelineno}{\thelinelineno.\arabic{linelinelineno}}
\newcommand{\nnnline}{\refstepcounter{lineno}%
\'\thelineno.\>\>\>%
}

% New line (3rd level) with referable label
\newcommand{\nnnlinelbl}[1]{\refstepcounter{lineno}\label{#1}%
\'\thelineno.\>\>\>%
}

% New line at 4th level
\newcommand{\nnnnline}{\refstepcounter{lineno}%
\'\thelineno.\>\>\>\>%
}

% New line (4th level) with referable label
\newcommand{\nnnnlinelbl}[1]{\refstepcounter{lineno}\label{#1}%
\'\thelineno.\>\>\>\>%
}




% Put a comment line in an algorithm, no line number.
\def\comment{\'\em }


%\settabs 17\columns

% Define the "assignment" symbol for algorithms
\newcommand{\asn}{\mbox{$\,\leftarrow$\,}}

%
% END OF DEFs FOR ALGORITHMS
%


\begin {document}
\begin{center}

\vspace{0.5in}

{\Large\em Elliptic Curves in Cryptography}

\vspace{0.25in}
{\large Ian Blake, Gadiel Seroussi and Nigel Smart}

\vspace{0.5in}

{\Large\bf Errata}

(for first and second printings\footnote{These errata have been corrected in the third printing of June 2000.})

\end{center}

\vspace{0.15in}

%\addtolength{\parskip}{5mm}

\begin{description}
\addtolength{\itemsep}{2mm}

\item[p.\ 7] Line 6 should read `finite fields and 
a finite number of abelian varieties.
For all practical purposes, the latter can be taken to be Jacobians of
curves.'

\item[p.\ 7] Line -3. Insert `some of' after `breaking'.

\item[p.\ 14] Step 6 of Algorithm II.2 (Barrett Reduction) should be
\[
 6. \;\;\mbox{\tt While  } z \geq p \;\;\;\mbox{\tt do  } z \asn z-p.  \]

\item[p.\ 42] Line 6 should read
\[ Y^2 + X Y = X^3 + a_2 X^2 + a_6. \]

\item[p.\ 57] Line -1 should read
\[ y_3 = (x_1-x_3) \lambda - y_1. \]

\item[p.\ 58] Line 4 should read
\[ y_3 = (x_1-x_3) \lambda - y_1. \]

\item[p.\ 76] Line -12 should read `So, $m$ can be `divided' by $\varphi^n-1$ ...'

\item[p.\ 89] Replace the first full paragraph with the following.

Let $E$ denote an elliptic curve defined over the field of $p$-adic
numbers, $\Q_p$, which is assumed to have good reduction at $p$.
The set of points of $E(\Q_p)$ which reduce to zero modulo $p$ is
denoted by $E_1(\Q_p)$ which is a group.
The set of points in $E(\Q_p)$ which reduce modulo $p$ to
an element of $E(\F_p)$ is denoted by $E_0(\Q_p)$.
In our case of $E$ having good reduction at $p$ we have 
$E(\Q_p) = E_0(\Q_p)$ but to remain consistent with the more general 
literature we shall still retain the notation $E_0(\Q_p)$.
There is the exact sequence

\item[pp.\ 92--97] {\bf Rho, lambda and kangaroo methods.}
Although the method described in Section V.5 is similar
to schemes recently used in practical attacks on the
ECDLP, our discussion of the rho, lambda and kangaroo methods
does not accurately reflect their original descriptions.
The following corrections are intended to clarify the differences
between the methods and reduce confusion.

\item[p.\ 92]
Replace the paragraph before the start of the example with the following.

The rho and lambda methods, discussed in more
detail in the next subsection, also have complexity $O(\sqrt{n})$.
The time to sort and search the look-up table in the BSGS method
can be eliminated if a hash table is used instead.
In this case, the constant multiplying $\sqrt{n}$ in the
asymptotic estimate can be made $\frac{4}{3}$. 
Pollard's rho method has a slightly better constant, roughly 
$\frac{5}{4}$. However, this is only an {\em expected} running time, 
given the randomized nature of the method. The lambda method, 
on the other hand, has a constant of 2, again applied to
expected running time.  The advantage of the rho and lambda methods is 
that their storage requirements can be made arbitrarily small.


\item[p.\ 93] {\bf Section V.5}. Change the title to 
{\bf `Methods based on Random Walks'}.

Replace the first two paragraphs of the section with the following.

Pollard [{\bf 125}] gives a number of methods to solve the discrete logarithm
problem in a variety of groups. The rho method uses a single
random walk and waits for a cycle to occur.  By using a space-efficient 
method to detect the cycle, the discrete logarithm can be
found.  The wait for the cycle means that the single random walk
can be thought of as tracing out the greek letter rho, $\rho$. 

In Pollard's lambda method (often called the method of tame and wild
kangaroos), two random walks are used, one by a tame kangaroo who jumps
off into the wild, digs a hole and waits for the wild kangaroo to
fall into it.  The two paths form the shape of the greek letter lambda,
$\lambda$. The lambda method is suited to finding 
discrete logarithms which are known to lie in a short interval.

There is a parallel version of the rho method, which uses many random
walks.  However, despite the method's name,
the `paths' do not now look like a rho, since instead one
looks for two paths that intersect.  The method described
in this section is what is usually referred to as the parallel rho 
method.

The following intuitive explanation uses the analogy of jumping animals,
since we have found this to be useful when explaining the method in
lectures.  However, these are not the kangaroos of Pollard's method, since
Pollard's kangaroos perform better controlled jumps. We shall call 
our jumping animals `snarks', since they jump around in 
a rather uncontrolled manner.

\item[p.\ 94] First paragraph should be replaced with the following.

To simplify the matter we take two snarks. Eventually we shall use 
a larger number of snarks.  The two snarks are given a spade and told that they
should dig a hole every ten or so jumps.  Where each snark jumps next
depends on the position they are currently at, hence when one snark
meets the path of the other (or itself) it will follow the original
path along until it falls into one of the holes that have been dug.


\item[pp.\ 94--96]
Replace `kangaroo' with `snark' throughout.  Remove references to `tame' and
`wild' snarks, since all snarks are `wild'.

\item[p.\ 95]
Line 17.  Remove `, which gives the method one of its names.'

\item[p.\ 97] 
Line -17. Remove the word `small'.

\item[p.\ 98] Line -6 should read `the other supersingular curves.'

\item[p.\ 104] Line -7. 
Replace `can be significantly diminished' with `can be made
arbitrarily small.'

\item[p.\ 112] Line -5 (of the main text). 
The inequality should be an equality.

\item[p.\ 114] Footnote. Insert comma after `comments'.

\item[p.\ 171] Line -11. The sum should run over $P \in C(\fbq)$.

\item[p.\ 176] Line 10. Remove `, what is no surprise,'.

\item[p.\ 193] In reference [37], the author is R.\ Crand{\em a}ll.


\item[pp.\ 191--198] The following references have now appeared.

[89]  {\em JLMS}, Vol 59, pp 448-460, 1999.

[118] {\em Math.\ Comp.}, Vol 68, pp 1233-1241, 1999.

[152] {\em J. Cryptology}, Vol 12, pp 141-151, 1999.

[153] {\em J. Cryptology}, Vol 12, pp 193-196, 1999. 

\end{description}

\end {document}