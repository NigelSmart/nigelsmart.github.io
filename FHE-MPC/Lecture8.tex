%%%%%%%%%%%%%%%%%%%%%%%%%%%%%%%%%%%%%%%%%%%%%%%%%%%%%%%%%%%%%%%%
%
%  Template for creating scribe notes 
%
%  Fill in your name, lecture number, lecture date and body
%  of scribe notes as indicated below.
%
%  Based on template used at Princeton University
%
%%%%%%%%%%%%%%%%%%%%%%%%%%%%%%%%%%%%%%%%%%%%%%%%%%%%%%%%%%%%%%%%


\documentclass[11pt]{article}

\usepackage{amsmath, amssymb}
\usepackage{amsthm}

\setlength{\topmargin}{0pt}
\setlength{\textheight}{9in}
\setlength{\headheight}{0pt}
\setlength{\headsep}{0pt}
\setlength{\oddsidemargin}{0.25in}
\setlength{\textwidth}{6in}
\pagestyle{plain}

\begin{document}

\thispagestyle{empty}

%%%% mark this version as a draft (remove for final revision) %%%%
%\raisebox{0.6in}[0in]{\makebox[\textwidth][r]{\it
 %DRAFT --- a final version will be posted shortly}}
%\vspace{-0.7in}
%%%%%%%%%%%%%%%%%%%%%%%%%%%%%%%%%%%%%%%%%%%%%%%%%%%%%%%%%%%%%%%%%%

\begin{center}
\bf\large FHE-MPC Notes
\end{center}

\noindent
Lecturer: Nigel Smart               %%% fill in lecturer
\hfill
Lecture \#8               %%% fill in lecture number here
\\
Scribe: Peter Scholl                 %%% fill in your name here
\hfill
28/11/11                         %%% fill in lecture date here

\noindent
\rule{\textwidth}{1pt}

\medskip


\newcommand{\mb}[1] {\mathbf{#1}}
\newcommand{\Norm}[1]{\Vert #1 \Vert}
\newcommand{\round}[1]{\lceil #1 \rfloor}

%%%%%%%%%%%%%%%%%%%%%%%%%%%%%%%%%%%%%%%%%%%%%%%%%%%%%%%%%%%%%%%%
%% body of scribe notes goes here
%%%%%%%%%%%%%%%%%%%%%%%%%%%%%%%%%%%%%%%%%%%%%%%%%%%%%%%%%%%%%%%%

In the previous lecture, it was shown how to perform secure multiparty
computation using a multiplicative Linear Secret Sharing Scheme (LSSS). The
resulting scheme had two main drawbacks:

\begin{itemize}
\item Perfect security only for \textit{semi-honest} adversaries.
\item Required secure point-to-point communication between all players.
\end{itemize}
To allow active as opposed to semi-honest adversaries can be 
dealt with via VSS, but this is inefficient and still does not 
avoid the secure point-to-point communication issue.

The requirement for secure channels stems from the algorithm for multiplying
$x$ and $y$: each player had to compute the local product of shares $p_i = [x]_i[y]_i$
and then securely distribute the shares of $p_i$ to all other players.
To allow the use of publicly broadcasted channels, a new technique for
multiplication is required. This was introduced by Beaver in 1992 \cite{Beaver1992}.

Initially, it is required that a secret \textit{multiplication triple} is
shared between all players. That is, the $i$-th player has the shares $[a]_i,
[b]_i, [c]_i$, such that upon reconstruction, $a \cdot b = c$.
Suppose we have already done this initial setup stage, and now want
to compute the product of $x$ and $y$, where each player
has the shares $[x]_i, [y]_i$. Firstly, every player computes and
publicly broadcasts the values:

\begin{align*}
[d]_i &= [x]_i - [a]_i \\
[e]_i &= [y]_i - [b]_i.
\end{align*}
Since $[a]_i$ and $[b]_i$ are only initially known to the $i$-th player, it
follows that these values hide the corresponding shares of $x$ and $y$. We also
have that when reconstructed, $d = x - a$ and $e = y - b$. Next, each player
reconstructs the values of $d$ and $e$, and then computes
\[[z]_i = de + d[b]_i + e[a]_i + [c]_i.\]
This corresponds to a share of
\begin{align*}
z &= de + db + ea + c \\
  &= (x-a)(y-b) + (x-a)b + (y-b)a + ab \\
  &= xy
\end{align*}
%
and so we have successfully shared the product of $x$ and $y$.


\subsection*{Active adversaries}
To achieve security against active adversaries, some kind of error correction
is needed. 
We need to be able to correct any errors introduced by the adversary in
the broadcast of the shares of $e$ and $d$.
In the case of information theoretic security this means that the
underlying LSSS must support a $Q^3$ adversary structure, or
equivalently can be extended to a scheme which is 
strongly multiplicative.
By a result of Cramer \textit{et al.} \cite{Cramer2000},
these properties are essentially the same as the LSSS being
derived from an error correcting code.
In the case of threshold adversaries this equates to the treshold
satisfying $t<n/3$.

For the case of dishonest majority and computational adversaries
a different technique is needed.
We could use MACs to provide authentication of the communications. 
Essentially, every piece
of transmitted data is augmented with a MAC to ensure that nobody can cheat.




\section*{Sharing a multiplication triple}

\subsection*{Using a threshold LSSS}

To generate a shared multiplication triple we can use pseudo-random secret
sharing (PRSS), as introduced in the previous lecture.
When using a Shamir secret sharing based scheme, the steps for each 
player to carry out are as follows:

\begin{itemize}
\item Using PRSS, generate shares $[a]_i$, $[b]_i$, $[r]_i$, of pseudo-random
numbers $[a]$, $[b]$ and $[r]$, which are represented by degree $t$ polynomials.

\item Use \textit{pseudo-random zero sharing} (PRZS) to generate shares $[z_i]$ of $0$,
represented by a degree $2t$ polynomial $z$.

\item Compute $[s]_i = [a]_i [b]_i - [r]_i + [z]_i$, giving a degree $2t$ share
of $ab - r$.

\item Broadcast $[s]_i$, so every player knows $s = ab - r$.

\item Compute the share $[c]_i = [s] + [r]_i$, giving a share of the product $[a]\cdot [b]$.

\end{itemize}

Note that, whilst the zero shares $[z]_i$ are not required for correctness, they
are necessary to properly mask $s$. Although the random polynomial $r$ provides
some masking, it is only of degree $t$ and so does not fully hide the degree
$2t$ polynomial.

In the case of semi-honest adversaries the above protocol will be correct.
For the case of active adversaries in the information theoretic model
we simply require that the honest players can detect when an error
occurs (since then we just abort the offline phase and can start again).
Since the broadcast of $[s]_i$ is of a $2t$ sharing we will be able
to detect errors as long as the number of honest players is larger
than $2t$; i.e. we require $n<t/3$.

\subsection*{Using somewhat homomorphic encryption}

We want to be able to perform multiparty computation when faced with a
\textit{dishonest majority} of adversaries. Moreover, we want these adversaries
to be \textit{active}.
We need to be able to produce the triples, and the associated MACs
mentioned above.
To overcome these obstacles, we make use of a \textit{somewhat homomorphic encryption}
scheme to compute the multiplication triples.
This avoids the dependency on
a strongly multiplicative LSSS, thus allowing security against a 
dishonest majority (at the expensive of assuming a computational assumption).
The steps for computing a multiplication triple using SHE are as follows
(see \cite{Damgard}).

\begin{itemize}
\item Generate random shares $[a]_i$ and $[b]_i$ as before (for dishonest
majority this is trivial).

\item Broadcast the \textit{encryptions} under the SHE scheme of these shares,
given by $\mathsf{Enc}([a]_i)$, $\mathsf{Enc}([b]_i)$.

\item Each player homomorphically computes $\mathsf{Enc}(a) = \sum_i \mathsf{Enc}([a]_i)$,
$\mathsf{Enc}(b) = \sum_i \mathsf{Enc}([b]_i)$ and $\mathsf{Enc}(c) = \mathsf{Enc}(a) \cdot \mathsf{Enc}(b)$.

\item Perform a threshold decryption of $\mathsf{Enc}(c)$,
so each player ends up with a share $[c]_i$ of the product $a\cdot b$.

\end{itemize}
How to add MACs to the above computation is explained in \cite{Damgard}.

\bibliographystyle{plain}
\begin{thebibliography}{1}

\bibitem{Beaver1992}
{\sc Beaver, D.}
\newblock {Efficient multiparty protocols using circuit randomization}.
\newblock {\em Advances in Cryptology - CRYPTO 1991\/} (1991).

\bibitem{Cramer2000}
{\sc Cramer, R.,
\sc Damg\r{a}rd, I.,
\sc Maurer, U.}
\newblock {General secure multi-party computation from any linear secret-sharing scheme}.
\newblock {\em Advances in Cryptology - EUROCRYPT 2000} (2000).

\bibitem{Damgard}
{\sc Damg\r{a}rd, I., Pastro, V., Smart, N., and Zakarias, S.}
\newblock {Multiparty Computation from Somewhat Homomorphic Encryption}.
\newblock {\em eprint.iacr.org\/} (2011).

\end{thebibliography}

\end{document}
