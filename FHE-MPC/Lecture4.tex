%%%%%%%%%%%%%%%%%%%%%%%%%%%%%%%%%%%%%%%%%%%%%%%%%%%%%%%%%%%%%%%%
%
%  Template for creating scribe notes 
%
%  Fill in your name, lecture number, lecture date and body
%  of scribe notes as indicated below.
%
%  Based on template used at Princeton University
%
%%%%%%%%%%%%%%%%%%%%%%%%%%%%%%%%%%%%%%%%%%%%%%%%%%%%%%%%%%%%%%%%


\documentclass[11pt]{article}

\usepackage{amsmath}
\usepackage{amsfonts}
\usepackage{amssymb}
\usepackage{url}
\newcommand{\F}{\mathbb{F}}
\newcommand{\Z}{\mathbb{Z}}
\newcommand{\Q}{\mathbb{Q}}
\renewcommand{\O}{\mathcal{O}}
\renewcommand{\i}{\mathfrak{i}}
\newcommand{\p}{\mathfrak{p}}
\newcommand{\Gal}{\textup{\textbf{Gal}}}

\setlength{\topmargin}{0pt}
\setlength{\textheight}{9in}
\setlength{\headheight}{0pt}
\setlength{\headsep}{0pt}
\setlength{\oddsidemargin}{0.25in}
\setlength{\textwidth}{6in}
\pagestyle{plain}

\begin{document}

\thispagestyle{empty}

%%%% mark this version as a draft (remove for final revision) %%%%
%\raisebox{0.6in}[0in]{\makebox[\textwidth][r]{\it
 %DRAFT --- a final version will be posted shortly}}
%\vspace{-0.7in}
%%%%%%%%%%%%%%%%%%%%%%%%%%%%%%%%%%%%%%%%%%%%%%%%%%%%%%%%%%%%%%%%%%

\begin{center}
\bf\large FHE-MPC Notes
\end{center}

\noindent
Lecturer: Nigel Smart                %%% fill in lecturer
\hfill
Lecture \# 3              %%% fill in lecture number here
\\
Scribe: David Bernhard                 %%% fill in your name here
\hfill
                         %%% fill in lecture date here

\noindent
\rule{\textwidth}{1pt}

\medskip

%%%%%%%%%%%%%%%%%%%%%%%%%%%%%%%%%%%%%%%%%%%%%%%%%%%%%%%%%%%%%%%%
%% body of scribe notes goes here
%%%%%%%%%%%%%%%%%%%%%%%%%%%%%%%%%%%%%%%%%%%%%%%%%%%%%%%%%%%%%%%%

\paragraph{Topics.}
Introduction to algebraic number theory and Galois theory; 
the mathematical background of the Gentry-Halevi-Smart and Smart-Vercauteren FHE schemes.

``Picking the right field'':
In ring-LWE the message space is $\F_2(X)/F(X)$, $R = \Z[X]/F(X)$. 
Over $\Q$, $F(X)$ is irreducible but over $\F_2$ probably not.

\paragraph{Algebraic Number Theory.}
Let $K = \Q[X] / F(X)$ where $F$ is an irreducible polynomial. 
Then $K$ is a field, it is called a \emph{number field}.
In $K$, there are many subrings for example $\Z[X] / F(X)$ which we can write as $\Z[\Theta]$ where $\Theta$ is a ``formal root''.
Then $K \cong \Q[\Theta]$.
There is a subring $\O_K$ satisfying $\Z[\Theta] \subseteq \O_K \subset K$, called the \emph{algebraic integers} and is the largest subring with certain nice properties. (The name comes from the fact that $\Z = \O_\Q$.)

Recall that an \emph{ideal} $\i$ in a ring $R$ is a set $\i \subseteq R$ such that for all $i_1, i_2 \in \i$ we also have $i_1 + i_2 \in \i$ and for all $i \in \i, r \in R$ we have $r.i \in \i$.
In $\O_K$ we have unique factorisation, that is for all ideals $\i$ we have $\i = \prod \p_i^{e_i}$ where the $\p_i$ are prime ideals and the $e_i$ integers.

Fact. For a prime ideal $\p$ of $\O_K$ we have $N(\p) = p^f$ where $N$ is the norm (number of elements in $R/\p$), $p$ is a prime number and $f$ an integer. 
In fact $\O_K/\p \cong \F_{p^f}$.
For example, taking $R = \Z$ and $\i = (3)$ we have $R/(3) = \F_3$.

\paragraph{Dedekind criterion.}
If $p \in \Z$ is a ``good prime'', that is 
$F(X) \equiv \prod_{i=1}^l F_i(X)\ \mod p$ where the $F_i$ are irreducible, then the ideal $\p = (p)$ factors as $\p = \p_1 \ldots \p_l$ and $R/\p_i = \F_p[X]/F_i(X)$.
(The CRT says that $R/\p = \prod_{i=1}^l R/\p_i$.)
We can write $\p_i = \{ p.r_1 + F_i(X).r_2 | r_1, r_2 \in R \}$ and abbreviate this to $\p_i = (p, F_i)$ which we call the \emph{two-element representation}.

(In the SV and GH FHE schemes, the secret key is some $\gamma \in R$ and the public key a two-element representation of $\gamma$.)

\paragraph{Galois groups.}
If $K = \Q(\Theta) = \Q[X]/F(X)$ and this contains all the $deg(F)$ roots of $F(X)$ then $K$ is Galois. 
In this case we have
 $(p) = \p_1^{e_1} \ldots \p_l^{e_l} \Rightarrow e_1 = e_2 = \ldots = e_l$ and $N(\p_1) = N(\p_2) = \ldots = N(\p_l)$.
Furthermore there is a Galois group 
$$\Gal(K/\Q) := \{ a \in \textup{Aut}(K) | a_{\downarrow \Q} = \textup{id}_\Q \}$$ 
which is a subset of the permutation group on the roots of $F(X)$.

Example. $F(X) = \Phi_m(X)$, a cyclotomic polynomial. 
Then $K$ is Galois and $\Phi_m = \prod F_i$, furthermore a $p$ is good if and only if $p \nmid m$.

The roots of $\Phi_m$ are $\zeta_m^{a_i} \in (\Z/m\Z)^*$.
The function $\kappa_{a_i}: x \mapsto x^{a_i}$ permutes these roots and in fact $\Gal(K/\Q) = \{ \kappa_{a_i} | a_i \in (\Z/m\Z)^* \}$.

Fact. All finite fields of the same size are isomorphic, in fact the only finite fields up to isomorphism are $\F_{p^d}$ where $p$ is prime and $d$ an integer.

\paragraph{Computing in finite fields.}
We wish to compute in $\F_{p^n} = \F_p[X]/G(X)$ where $deg(G) = n$.
(For example, in AES we have $p = 2$ and $G(X) = X^8 + X^4 + X^3 + X + 1$.)
For $F(X) = \Phi_m(X)$ the plaintext space will be $\Z[\Theta]\ \mod p$ which is isomorphic to $\prod_{i=1}^l \F_p[X]/F_i(X)$.

Fact. If $K = \Q[X] / \Phi_m(X)$ then $\O_K = \Z[\Theta]$.

If $a(\theta) \mod (p, F)$ is mapped under this isomorphism we wnd up with a vector 
$$\big(a_1(\Theta) \mod (p, F_1(\Theta)), \quad \ldots, \quad a_l(\Theta) \mod (p, F_l(\Theta))\big)$$
If we are careful in the values we pick we get $F = \Phi_m$ of degree $d$ and
$\Z[\Theta] \mod p \cong (\F_{p^d})^l$. 
If $n | d$ then $F_{p^n} \subset F_{p^d}$ so in fact we have $(\F_{p^n})^l \subset (\F_{p^d})^l$ and these maps are efficient so we can work with $l$-vectors of plaintexts at once.

\subparagraph{A global view.}
Taking $\Q[X] / F(X)$ as a degree $n$ Galois extension of $\Q$, the Galois group is a transitive group of permutations on the roots, i.e. for all $1 \leq i < j < n$ there is a $\sigma \in \Gal$ such that $\sigma.r_i = r_j$ (where $r_i, r_j$ are the $i$-th and $j$-th roots).

\subparagraph{A local view.}
Looking at $\F_p[X] / F_i(X)$ as a degree-$d$ extension of $\F_p$ we get $\Gal \cong C_d$, the cyclic group of order $d$. 
It has as generator the Frobenius map $x \mapsto x^p$.

\subparagraph{Combining the views.}
If $F = \Phi_m$ then $\Gal(K/\Q)$ contains the Frobenius map.
It will permute the roots in each subclass induced by a $F_i$ but not move them between these subclasses.
So what is a map that moves roots from one subclass to another?
Exactly $d$ of the maps of form $x \mapsto x^i$ are of the form $x \mapsto x^{p^d}$ as $p^d \equiv 1 \mod m$.
So $\Gal(K/\Q)$ contains a group generated by $p$, called the \emph{decomposition at} $p$ and written $G_p$.
Consider the group $H = \Gal/G_p$.

\paragraph{Examples}

\subparagraph{Example 1}
Let $m = 11$ and $p = 23$.
Then $\Phi_m(X) = (X-r_0)\ldots(X-r_9)$ splits into linear factors.
This gives us $10$ copies of $\F_{23}$ with componentwise addition and multiplication.
To move components around we note that $p^d = 22 \equiv 0 \mod m$ and $G_p = (1)$ so $\Gal/G_p = \Gal$.
By transitivity there must be a map that takes each component to each other one.

Suppose we have two vectors $v$ and $w$ and want to compute $v_1 + w_9$.
We can multiply $v$ with $(1, 0, \ldots 0)$,
apply a permutation to $w$ that brings $w_9$ into the first component then multiply this with $(1, 0, \ldots, 0)$ too and add the two resulting vectors to get $(v_1 + w_9, 0, \ldots, 0)$.
We know that we can add and multiply homomorphically (on ciphertexts) so we only need a way to compute the permutation homomorphically.

\subparagraph{Example 2}
Let $m = 31$ and $p = 2$.
We find $2^5 \equiv 1 \mod m$ so $d = 5$.
$\Phi_m(X)$ has $6$ factors of degree $5$ each and $\Gal \cong <2> \times <6>$.
Note that $\Gal / <2> = <6> \subset \Gal$.
If we pick the $F_i$ such that $F_i(x^{6^i}) \equiv 0 \mod F_1(X)$ then $\sigma_6: x \mapsto x^6$ moves $(m_0, \ldots, m_5) \mapsto (m_5, m_0, \ldots, m_4)$ and from this rotation we can get all others.
The inverse of $\sigma_6$ which we could call $\sigma_{1/6}$ is $(\sigma_6)^5$ because $(\sigma_6)^6 \equiv 1 \mod m$.

\subparagraph{Example 3}
Let $m = 257$ and $p = 2$.
Then $m | (2^{16} - 1)$ and so $d = 16$.
$H = \Gal/<2>$ has $16$ elements and is generated by a coset of $3$ as $3^8 \equiv 136 \mod m$ which is not an element of $<2>$ although $3^{16} \equiv 249 \equiv 2^{12} \mod m$.
We can compute that 
$$\sigma_3 . (m_0, \ldots, m_{15}) = ((m_{15})^{2^{11}}, m_0, m_1, \ldots, m_{14})$$
Similarly 
$$\sigma_{1/3} . (m_0, m_1, \ldots, m_{15}) = (m_1, m_2, \ldots, m_{15}, (m_0)^{32})$$

We can still write every permutation $\sigma$ as a sum of terms of ``basis'' vectors (with one $1$ and the rest zeroes) and permuations $\sigma_i$.
However, this can be computed more efficiently using permutation networks.

Note that if we consider $(\F_2)^l \hookrightarrow (\F_{2^d})^l$ then $m_j \mapsto (m_j)^{2^k} \equiv m_j \mod 2$ so the extra exponents disappear $\mod 2$.

Finally, consider a polynomial $\alpha = a_0 + a_1 X + \ldots + a_{n-1} X^{n-1}$ over $\F_{p^n}$.
We are interested in ``projecting'' out a coefficient.
There is a matrix $A$ such that $A (\alpha, \sigma_p.\alpha, \ldots, \sigma_{p^{n-1}} . \alpha)^T = (a_0, a_1, \ldots, a_{n-1})$ which will do the job for us.
This process can even be ``vectorised'' so over $\F_{2^8}$, the map $(a_0, \ldots, a_n) \mapsto (a_0^{\phantom{0}254}, \ldots, a_n^{\phantom{n}254})$ can be computed in only $3$ significant operations.

\paragraph{Further reading.}
More information on the theory we have covered (and related topics) seems to be available at \url{http://wstein.org/books/ant/ant/ant.html}.

%%%%%%%%%%%%%%%%%%%%%%%%%%%%%%%%%%%%%%%%%%%%%%%%%%%%%%%%%%%%%%%%

\end{document}
