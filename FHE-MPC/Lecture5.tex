%%%%%%%%%%%%%%%%%%%%%%%%%%%%%%%%%%%%%%%%%%%%%%%%%%%%%%%%%%%%%%%%
%
%  Template for creating scribe notes 
%
%  Fill in your name, lecture number, lecture date and body
%  of scribe notes as indicated below.
%
%  Based on template used at Princeton University
%
%%%%%%%%%%%%%%%%%%%%%%%%%%%%%%%%%%%%%%%%%%%%%%%%%%%%%%%%%%%%%%%%


\documentclass[11pt]{article}
\usepackage{amsmath,amsthm,amsfonts,color,url}

\newcommand{\red}[1]{\textcolor{red}{#1}}
\newcommand{\blue}[1]{\textcolor{blue}{#1}}
\newcommand{\green}[1]{\textcolor{green}{#1}}

\newcommand{\F}{{\mathbb F}}
\newcommand{\Q}{{\mathbb Q}}
\newcommand{\R}{{\mathbb R}}
\newcommand{\Z}{{\mathbb Z}}

\newcommand{\Enc}{{\sf Enc}}
\newcommand{\cnew}{{\mathfrak c}}
\newcommand{\round}[1]{\lfloor #1 \rceil}


\setlength{\topmargin}{0pt}
\setlength{\textheight}{9in}
\setlength{\headheight}{0pt}
\setlength{\headsep}{0pt}
\setlength{\oddsidemargin}{0.25in}
\setlength{\textwidth}{6in}
\pagestyle{plain}

\begin{document}

\thispagestyle{empty}

%%%% mark this version as a draft (remove for final revision) %%%%
%\raisebox{0.6in}[0in]{\makebox[\textwidth][r]{\it
 %DRAFT --- a final version will be posted shortly}}
%\vspace{-0.7in}
%%%%%%%%%%%%%%%%%%%%%%%%%%%%%%%%%%%%%%%%%%%%%%%%%%%%%%%%%%%%%%%%%%

\begin{center}
\bf\large FHE-MPC Notes
\end{center}

\noindent
Lecturer: Nigel Smart    %%% fill in lecturer
\hfill
Lecture \#      5        %%% fill in lecture number here
\\
Scribe:   Jake Loftus    %%% fill in your name here
\hfill
                         %%% fill in lecture date here

\noindent
\rule{\textwidth}{1pt}

\medskip

%%%%%%%%%%%%%%%%%%%%%%%%%%%%%%%%%%%%%%%%%%%%%%%%%%%%%%%%%%%%%%%%
%% body of scribe notes goes here
%%%%%%%%%%%%%%%%%%%%%%%%%%%%%%%%%%%%%%%%%%%%%%%%%%%%%%%%%%%%%%%%


\section{Modular reduction}
For an integer $t$ we let $[t]_q$ denote the reduction $t \pmod{q}$ into the interval $[-q/2,q/2)$.  This can be computed as $t-q\cdot\round{t/q}$ (e.g $15 \pmod{7} = 1 = 15 - 7 \cdot\round{15/7}$ and $8 \pmod{5} = 8 - 5\cdot\round{8/5} = -2$).  Suppose $q$ is an odd modulus and $t$ is an integer.  Then we can compute $[t]_q \pmod{2}$ as
\[
\begin{split}
[t]_q & = t - q\cdot\round{t/q} \pmod{2}\\
      & = t - \round{t/q} \pmod{2}
\end{split}
\] 

Now express $t/q$ in binary expansion \textit{i.e} $t/q = \Sigma_{i = - \infty}^{\infty} e_i\cdot 2^i$ where all but finitely many of the $e_i$ are zero.  If we round this and take the result $\pmod{2}$ then in fact the result is simply the xor of the bits either side of the decimal point \textit{i.e} $[t]_q \pmod{2} = e_0 \oplus e_{-1}$.  This can easily be seen by writing out a few examples.

Suppose now we wish to compute the sum of $l$ integers $[\Sigma t_i]_q \pmod{2}$.  First compute a big table of binary expansions as 
\[ \begin{array}{lc}
t_1/q  & \ldots 11.01001 \ldots \\
t_2/q  & \ldots 01.10011 \ldots \\
\vdots & \vdots                 \\ 
t_l/q  & \ldots 10.11010 \ldots   
\end{array}\]

Then because of carries we can't just xor things and add.  We actually need to retain the approximately first $log_2(l)$ bits.  This addition and rounding is essentially the non-linear part of the bootstrapping operation.



\section{Bootstrapping}
The key idea behind bootstrapping is to evaluate the decryption circuit homomorphically resulting in a clean encryption of the original message.  In the original SV/GH schemes \cite{SV10}, \cite{GH10} this was achieved by augmenting the public key with some additional information, namely integers $\{ x_i \}_{i=0}^{i=S}$ such that $s$ of those integers add up to the secret key $w$ for the initial somewhat homomorphic encryption scheme with $s << S$.  The secret key $\{\sigma_i \}_{i=0}^{i=S}$ is now the characteristic vector of that sparse subset.  We also include encryptions $\cnew_i = \Enc(\sigma_i,pk)$ of the new secret key bits in the public key. Note $w = \Sigma_{0}^{S} \sigma_i\cdot x_i \pmod{d}$.

Bootstrapping then procedes in the following stages:
\begin{itemize}
\item Write down a matrix of $s$ by $p = \lceil log_2(s+1) \rceil $ bits $\{ b_{i,j} \}$ which correspond to the first $p$ bits in the binary expansion of $cx_i / d $ (similar to above, now as a matrix.

\item Encrypt each of these bits to obtain \textit{clean} (\textit{ie} with small noise) ciphertexts $c_{i,j}$.

\item Multiply each row of this matrix by the corresponding encryption $\cnew_i$ of the secret key bits to obtain a matrix $\{ \cnew_i \cdot c_{i,j} \pmod{d} \}$.

\item Now we need to compute the encryption of the sum of the plaintext bits $\sigma_i \cdot b_{i,j}$ in each of the columns separately. 
  \begin{itemize}
  \item Labeling in reverse corresponding to lower bits (as above), for column $-j$ we compute the carry bit to be sent to column $-j+t$ as the elemmentary symmetric polynomial $\pmod{2}$ of degree $2^t$ in the bits of column $-j$.  This is just the $t'th$ bit of the hamming weight of that column. 
  \item Form a suitable matrix and use carry-save-adders (see (\cite{wiki})) in \cite{SV10} or "grade-school" addition in \cite{GH10} to reduce it to a matrix with two rows.
   \end{itemize}
\item In the final stage we need to xor the two remaining encrypted bits to obtain the clean encryption of the original message.

\end{itemize}

This performs the function required even if it does seem a little cumbersome.  In particular bootstrapping is possible if we can evaluate elementary symmetric polynomial up to a certain degree in in \cite{SV10} and Gentry's original scheme or if we can use the "school-book" addition method as found in \cite{GH10}.

Recall in \cite{BGV11} for the RLWE variant, ciphertexts are vectors of elements of a ring $R_q = \Z_q[x] / (f) $.  After key switching a ciphertext will be of the form $(c_0,c_1)$ and decryption can be computed as $[c_0 - s\cdot c_1]_q \pmod{2}$.  Let $s = \Sigma s_i X^i, c_0 = \Sigma u_i X^i$ and $c_1 = \Sigma v_i X^i$. Then the $i'th$ coefficient of $c_o - s \cdot c_1$ is given as $\Sigma s_j \cdot w_j + w_{-1}$ where $w_{-1}$ is the additional term appearing due to reduction by the field polynomial.  In this case we then represent the bits of each $w_j / d$ in a matrix and apply the same method as above to bootstrap (we assume the coefficients of the secret key are in $\{0,1 \}$ for simplicity - other keys can easily be dealt with).  Note in particular the abscense of any sparse subset sum problems.

\begin{thebibliography}{xx}

\bibitem{BGV11}
Z. Brakerski, C. Gentry and V. Vaikuntanathan
\newblock Fully Homomorphic Encryption without Bootstrapping
\newblock To appear in Innovations in Theoretical Computer Science 2012.


\bibitem{GH10}
C. Gentry and S. Halevi.
\newblock Implementing Gentry’s Fully-Homomorphic Encryption Scheme. 
\newblock In {\em EUROCRYPT 2011}, volume 6632 of
Lecture Notes in Computer Science, pages 129–148. Springer, 2011.


\bibitem{SV10}
N.P. Smart and F. Vercauteren.
\newblock Fully homomorphic encryption with relatively small key and ciphertext sizes. 
\newblock In {\em Public Key Cryptography -- PKC 2010}, 
Springer LNCS 6056, 420--443, 2010

\bibitem{wiki}
\url{http://en.wikipedia.org/wiki/Carry-save_adder}

\end{thebibliography}





%%%%%%%%%%%%%%%%%%%%%%%%%%%%%%%%%%%%%%%%%%%%%%%%%%%%%%%%%%%%%%%%

\end{document}
